\documentclass[11pt]{article}

\usepackage{amsthm,amsmath,amssymb,amsfonts}
\usepackage[margin=1in]{geometry}

\parindent 0pt
\parskip 3mm

\theoremstyle{definition}
\newtheorem*{solution}{Solution}

\renewcommand{\pmod}[1]{\mbox{\ $(\mathrm{mod}\ {#1})$}}
\providecommand{\Leg}[2]{\genfrac{(}{)}{}{}{#1}{#2}}
\newcommand{\Z}{\mathbb{Z}}

\begin{document}

\begin{center}
{\bf \Large CPSC 418 / MATH 318 --- Introduction to Cryptography

ASSIGNMENT 3 \qquad Problems 6-7}
\end{center}

\hrule 	

\textbf{Name:} Crypto Wizard (replace by your name) \\
\textbf{Student ID:} 0000000 (replace by your ID number)

\medskip \hrule

\begin{enumerate} \itemsep 20pt

\item[] \textbf{Problem 6} --- A probabilistic encryption scheme (28 marks)


\begin{enumerate}

\item \begin{enumerate}

\item % 6 (a) (i)

\item % 6 (a) (ii) 

\item % 6 (a) (iii)

\item % 6 (a) (iv)

\end{enumerate}

\item % 6 (b)

\item % 6 (c)

\item % 6 (d)

\item % 6 (e)
\end{enumerate}

\newpage

\item[] \textbf{Problem 7} --- Universal forgery against improper El Gamal signatures, 12 marks)

\begin{enumerate}

\item % 7 (a)

\item % 7 (b)

\item % 7 (c)

\end{enumerate}

\end{enumerate}

\end{document}
