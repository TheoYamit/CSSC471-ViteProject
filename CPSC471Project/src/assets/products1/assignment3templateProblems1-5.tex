\documentclass[11pt]{article}

\usepackage{amsthm,amsmath,amssymb,amsfonts}
\usepackage[margin=1in]{geometry}

\parindent 0pt
\parskip 3mm

\theoremstyle{definition}
\newtheorem*{solution}{Solution}

\renewcommand{\pmod}[1]{\mbox{\ $(\mathrm{mod}\ {#1})$}}
\providecommand{\Leg}[2]{\genfrac{(}{)}{}{}{#1}{#2}}

\newcommand{\CBCMAC}{\text{CBC-MAC}}
\newcommand{\AHMAC}{\mathrm{AHMAC}}
\newcommand{\PHMAC}{\mathrm{PHMAC}}

\begin{document}

\begin{center}
{\bf \Large CPSC 418 / MATH 318 --- Introduction to Cryptography

ASSIGNMENT 3 \qquad Problems 1-5}
\end{center}

\hrule 	

\textbf{Name:} Theodore Yamit \\
\textbf{Student ID:} 30141383 (replace by your ID number)

\medskip \hrule

\begin{enumerate} \itemsep 20pt

\item[] \textbf{Problem 1} --- Pre-image resistance versus collision resistance (14 marks)

\begin{enumerate}

\item \begin{enumerate}

\item From the question, we know that h is pre-image resistant, such that given a hash $x$ and $H(M) = x$, it is computationally infeasible to find $M$. We now want to prove that the new hash function $H$ is also pre-image resistant.\\\\
\textbf{Proof by Contradiction:}\\
Let's assume that $H$ is NOT pre-image resistant.\\
This means that we can find a pre-image of $x$, such that we can find $M$ for $H(M) = x$.\\

Since we are able to find a pre-image of $x$ under $H$, and $H(M) = h(M')$, then this means we are also able to find a pre-image under $h$. However, since $h$ is a pre-image resistant has function, then this leads to a contradiction (since if $H$ is no pre-image resistant, then $h$ must also not be pre-image resistant).\\

Thus, our initial assumption of $H$ not being pre-image resistant must be wrong, making $H$ pre-image resistant.\\

\item To prove that $H$ is not collision resistant, we can provide a counter example.\\
Note: We will be proving this for strong collisions (since this will prove $H$ is not weak-collision resistant by definition as well).\\

Since we a proving for strong collision resistance, it is sufficient to show that it is computationally infeasible to find two distinct $M_1$ and $M_2$ such that $H(M_1) = H(M_2)$.\\

\textbf{Counterexample:}\\
Let $M_1$ equal any bit string of length $n$, but have the last bit be a 0.\\
Let $M_2$ be another unique bit string of length $n$, where the first $n-1$ bits are the same $n-1$ bits of $M_1$, except for the last bit, which is a 1. From the description of this question, $M'$ agrees with $M$, except for the last bit, which is changed from 1 to a 0 (in the case it is 1).\\

Thus, we have:
\begin{align*}
M_1' &= M_1\\
M_2' &= M_2 \text{ (Where the last bit of } M_2 \text{ is changed from a 1 to a 0)}
\end{align*}

We can see that $M_1' = M_2'$, since the first $n-1$ bits of both messages are the same, and $M_2$'s last bit was changed from a 1 to a 0, matching $M_1$.\\\\
Since $M_1' = M_2'$, we have $h(M_1) = h(M_2)$, and thus, $H(M_1) = H(M_2)$.\\\\
Since we have found a collision, and $M_1 \ne M_2$, we have proven that $H$ is not a strongly collision resistant hash function (which also makes it not a weakly collision resistant hash function either).\\
\end{enumerate}

\item \begin{enumerate}

\item  We want to prove that $H$ is collision resistant (strongly collision resistant, since this proves it is also weakly collision resistant). $H$ being strongly collision resistant means that it is computationally infeasiable to find two distinct messages $M_1$ and $M_2$, such that $H(M_1) = H(M_2)$.\\

\textbf{Proof by Contradiction:}
Let's assume that H is NOT collision resistant.\\
This means that it is computationally feasible to find two messages $M_1$ and $M_2$, where $M_1 \ne M_2$, such that $H(M_1) = H(M_2)$.\\

Suppose we have two messages $M_1$ and $M_2$.\\

There are two cases to consider such that they land on the same case (since one case prepends 0, and the other cases prepends 1):\\

1. When $M_1$ and $M_2$ are both 0 (both are just the 0 bit).\\
2. When $M_1$ and $M_2$ are both distinct and not just the 0 bit.\\

\textbf{1st case (Both $M_1$ and $M_2$ are both the 0 bit):}\\
While this results in the same hash of $1 || 0^n$, which showcases a collision ($H(M_1) = H(M_2)$), $M_1 = M_2$, meaning they aren't distinct messages. This leads to a contradiction. \\

\textbf{2nd case (Both $M_1$ and $M_2$ are both distinct and not just the 0 bit):}\\
Then $H(M_1) = 0||h(M_1)$ and $H(M_2) = 0||h(M_2)$. Since $M_1 \ne M_2$, we want it such that there is a collision for $h(M_1)$ and $h(M_2)$, such that $h(M_1) = h(M_2)$. However, since $h$ is a collision resistant hash function, this can't happen, and thus we have another contradiction.\\

Since both cases lead to a contradiction, our initial assumption of $H$ not being collision resistant must be wrong, and thus, $H$ is a collision resistant hash function.\\ 

\item Pre-image resistance means that given a hash $x$, such that $H(M) = x$, it is computationally infeasible to find $M$. So we can make a counterexample with a specific $x$.\\

Thus, given a hash $x$, such that $H(M) = x$, suppose that $x$'s string starts with 1. Then we instantly obtain $M$. This is because for $x$ to start with 1, $M$ must be the 0 bit (since 1 is prepended in this case, while 0 is prepended if $M$ is not the 0 bit). Since it is not computationally infeasible in this case, where the hash $x$ starts with 1, then it is possible to find $M$ such that $H(M) = x$ (namely, $M = 0$), which proves that $H$ is not pre-image resistant.
\end{enumerate}

\end{enumerate}

\newpage

\item[] \textbf{Problem 2} --- CBC-MAC and one-register CFB-MAC (6 marks)

\begin{enumerate}

\item First, let's state definitions from both CBC-MAC and CFB-MAC.\\
For CBC-MAC:
\begin{align}
C_0 = 0^n &\text{ (Initial State)}\\
C_i = E_K(M_i \oplus C_{i-1}) &\text{ (For } 1 \le i \le L \text{)}\\
\text{CBC-MAC}(M) = C_L &\text{ (Final State)}
\end{align}
For CFB-MAC:
\begin{align}
C_0'= M_1 &\text{ (Initial State)}\\
C_i'= M_{i+1} \oplus E_K(C_{i-1}') &\text{ (For } 1 \le i \le L - 1 \text{)}\\
\text{CFB-MAC}(M) = E_K(C_{L-1}') &\text{ (Final State) }
\end{align}

Let's use induction on $i$ to prove that $C_i = C_i' \oplus M_{i+1}$ for $0 \le i \le L - 1$.\\

\textbf{Base Case - i = 0:}
\begin{align*}
C_0 &= C_0' \oplus M_{0+1}  &\textbf{By our assumption}\\
C_0 &= M_1 \oplus M_1  &\textbf{By definition of CFB-MAC (5)}\\
C_0 &= 0^n &\textbf{XOR laws: } \mathbf{A \oplus A = 0^n : (n = |A|)}
\end{align*}
Since $C_0 = 0^n$, by definition in (1), $C_i = C_i' \oplus M_{i+1}$  is true for the base case.\\

\textbf{Inductive Hypothesis: }Suppose that $C_k = C_k' \oplus M_{k+1}$ holds true for some $k$, where $0 \le k \le L - 1$.

\textbf{Inductive Step: }Thus, it is sufficient to show that $C_{k+1} = C_{k+1}' \oplus M_{k+2}$ holds true for $k+1$, where $0 \le k \le k + 1 \le L - 1$.\\\\
Note: k is different from K (K is the key and not really apart of the proof)\\

\textbf{Proof by induction: }
\begin{align*}
C_{k+1} &= E_K(M_{k+1} \oplus C_k) &\textbf{By definition for CBC-MAC (2)}\\
C_{k+1} &= E_K(M_{k+1} \oplus C_k' \oplus M_{k+1}) &\textbf{By our inductive hypothesis}\\
C_{k+1} &= E_K(M_{k+1} \oplus M_{k+1} \oplus C_k') &\textbf{Property of XOR: Commutative}\\
C_{k+1} &= E_K(C_k') &\textbf{XOR rule: A } \mathbf{\oplus} \textbf{ A = 0}
\end{align*}
Now,
\begin{align*}
C_{k+1}' &= M_{k+2} \oplus E_K(C_{k}') &\textbf{By definition for CFB-MAC (5)}\\
C_{k+1}' &= M_{k+2} \oplus C_{k+1} &\textbf{From previous step } \mathbf{C_{k+1} = E_K(C_k')}\\
C_{k+1} &= C_{k+1}' \oplus M_{k+2} &\textbf{XOR property: }\mathbf{A = B \oplus C \rightarrow C = A \oplus B}
\end{align*}
We have shown that if this statement holds for $k$, then this statement also holds for $k+1$, proving our inductive hypothesis.\\

Thus, using induction on i, we have proven that $C_i = C_i' \oplus M_{i+1}$ for $0 \le i \le L -1$. $\blacksquare$\\

\item Let M be any message, and K be a given key. \\
We want to show that CBC-MAC(M) = CFB-MAC(M).\\

\textbf{Direct proof: }\\
For $i = L - 1$ (the second last block), we have:
\begin{align*}
C_{L-1} &= C_{L-1}' \oplus M_{L} &\textbf{By our inductive proof in part (a)}
\end{align*}
For $i = L$ (The last block), we have:
\begin{align*}
\text{CBC-MAC} (M) &= C_L &\textbf{By definition of CBC-MAC (3)}\\
&= E_K(M_L \oplus C_{L-1}) &\textbf{By definition of CBC-MAC (2)}\\
&= E_K(M_L \oplus C_{L-1}' \oplus M_L) &\textbf{From our previous step on } \mathbf{C_{L-1}}\\
&= E_K(M_L \oplus M_L \oplus  C_{L-1}') &\textbf{Property of XOR: Commutative}\\
&= E_K(C_{L-1}') &\textbf{XOR rule: A } \mathbf{\oplus} \textbf{ A = 0}\\
\text{CBC-MAC}(M) &= \text{CFB-MAC}(M) &\textbf{By definition of CFB-MAC (6)}
\end{align*}
Thus, we have proven that CBC-MAC$(M)$ = CFB-MAC$(M)$. $\blacksquare$

\end{enumerate}

\newpage

\item[] \textbf{Problem 3} --- Flawed MAC designs (14 marks)

\begin{enumerate}

\item 
Let's list what we know so far:\\
- The attacker has a message/PHMAC pair ($M_1$, PHMAC$_K(M_1)$)\\
- $X$ is an $n$-bit block and $M_2 = M_1||X$
- We want to compute PHMAC$_K(M_2)$ without the key $K$.\\

Notice that $M_2 = M_1||X$, or more specifically, that $M_2$ starts with $M_1$.\\

From the ITHASH function, we know that this is an iterated hash function that works on a message $M$, consisting of $L$ blocks.\\

We are given a message/PHMAC pair ($M_1$, PHMAC$_K(M_1)$). We can see from the ITHASH function, when the input is $M_1$, such that PHMAC$_K(M_1)$, the returned PHMAC$_K(M_1)$ is the final state when $i = L$.\\

We are given PHMAC$_K(M_1)$ already, which would have already have had the key $K$ appended at the start of the ITHASH algorithm. Since we don't have the key, we can just start the algorithm ITHASH again but starting from the final state of PHMAC$_K(M_1)$, and then just continue processing from the block $X$ (Which makes sense because $M_2 = M_1||X$).\\


\item % 3 (b)

\end{enumerate}

\newpage

\item[] \textbf{Problem 4} --- El Gamal is not semantically secure (12 marks)

\newpage

\item[] \textbf{Problem 5} --- An IND-CPA, but not IND-CCA secure version of RSA (12 marks)

\end{enumerate}

\end{document}
